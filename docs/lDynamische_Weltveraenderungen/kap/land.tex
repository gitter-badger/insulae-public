\chapter{Landver{\"a}nderungen}
Unter den Landver{\"a}nderungen werden alle Szenarien zusammen gefasst, welche sich
auf das optische Erscheinungsbild der Karte auswirken.\newline
\newline
Nach aktuellem Stand bietet sich jedes der hier vorgestellten Szenarien auch an
um das Eingreifen der G{\"o}tter in die Welt darzustellen. Es ist noch zu
pr{\"u}fen inwieweit sich einige dieser Szenarien eignen durch m{\"a}chtige
Zauber durch Spieler bewu\ss t herbeigef{\"u} hrt werden k{\"o}nnen.

\section {Vulkanausbr{\"u}che}
Vulkanausbr{\"u}che dienen dazu nachhaltig die Landmasse zu ver{\"a}ndern.
Aufgrund ihrer grossen Zerst{\"o}rungskraft welche potentiell s{\"a}mtliche Geb{\"a}ude
und Landschaftsaufwertungen vernichten kann, ist dieses Ereigniss nur {\"a}usserst
selten zu triggern.
\linebreak\linebreak
Bei Verwendung von Vulkanausbr{\"u}chen ist zu Beginn jeder Runde klar zu stellen
das es zu solchen kommen kann mit einem Hinweis darauf was fuer endgueltige
Folgen ein Vulkanausbruch mit sich bringt. Das Gebiet in direkter N{\"a}he des
Vulkanes wird unmittelbar zerstort. Daraufliegende Geb{\"a}ude und Stra\ss en
werden ebenfalls unwiederbringlich zerst{\"o}rt. Es findet kein materieller oder
finanzieller Ausgleich statt. Charaktere welche zum Zeitpunkt des Ausbruches
sich in dem Gebiet aufhalten, werden zum n{\"a}chsten Tempel transferiert.
Sollte sich in der Region des Ereignisses ein Tempel befinden, ist dieser mit
Beginn des Ereignisses zu deaktivieren und sterbende Charaktere auf umliegende
Tempel zu verteilen.\linebreak

Nach dem Ausbruch gibt es mehrere Varianten wie das Ereignis fortgef{\"u}hrt
werden kann.\newline
- Ende des Ausbruches\newline
- Lava fliesst gen K{\"u}ste\newline
- Vernichtung der Insel\newline

Bei Variante 1, Ende des Ausbruches ist weiter nichts mehr zu beachten.  Die
vernichteten L{\"a}ndereien werden als Steinfelder fortgef{\"u}hrt welche wie
gehabt von Charakteren bewirtschaftet werden koennen. Das zentrale Feld besitzt
eine erh{\"o}hte Wahrscheinlichkeit Quelle von Edelsteinen zu sein.\newline
\linebreak
Sollte die Lava wie in Variante 2 gen K{\"u}ste flie\ss en, muss der Vektor zu
Beginn festgelegt werden. Von Vulkankrater bis zur K{\"u}ste wird ein 3-Felder
breiter Streifen Land vollst{\"a}ndig zerst{\"o}rt. Der Lavastrom breitet sich
mit mindestens einem Feld pro Tag aus. Schnelleres \glqq flie\ss en\grqq  der
Lava ist m{\"o}glich bis zu einer Rate von einem Feld pro Stunde.\newline
\linebreak
Die vollst{\"a}ndige Vernichtung einer Insel schlie\ss lich ist vergleichbar mit
Variante 2. Der Unterschied besteht lediglich darin das die Lava in alle
Richtungen in sich konzentrisch um den Startpunkt ausbreitet. Diese Variante ist
nur zu empfehlen wenn eine Insel vergr{\"o}\ss ert oder gegen eine andere
Inselvorlage ausgetauscht werden soll. In diesem Fall wird nachdem die Lava
erkaltet ist, anstelle der bisherigen Insel die neue entstehen. Ich empfehle um
einen glaubhaften {\"U}ebergang zu gestalten eine Kombination aus Vulkanausbruch
und {\"U}berflutung. Bei der {\"U}berflutung ist jedoch die {\"u}bliche Eintragung
in der {\"o}ffentlich zug{\"a}nglichen Historie auszublenden.\newline
\newline
Vulkanausbr{\"u}che k{\"o}nnen in einer laufenden Runde auch genutzt werden um
die Landmasse der Spielerzahl anzupassen. Gleichzeitig dienen sie dazu den
Eindruck einer ver{\"a}nderlichen Welt darzustellen.\newline
\newline
\textbf{Skriptname:} ringoffire.py
\newline\newline
\textbf{Aufrufparameter:} Startkoordinate, Variante, Geschwindigkeit,
Zielkoordinate, InselID\newline \newline
Werden einer oder mehrere der Parameter nicht gesetzt, wird von ringoffire.py
ein zuf{\"a}lliger Wert zugewiesen. Die Parameter Geschwindigkeit, Zielkoordinate
und InselID werden in Abh{\"a}ngigkeit der angegebenen Variante ausgewertet.

\section {{\"U}berflutungen}
Eine {\"U}berflutung ist in ihrer Auswirkung das Gegenteil zu
Vulkanausbr{\"u}chen. Anstatt Land zu schaffen, wird
hier Land genommen.\newline Daher treffen die oben genannten Anmerkungen in
verleichbarer Form auch auf {\"U}berflutungen zu. So kann eine {\"U}berflutung ebenfalls dazu genutzt werden die Landmasse der Spielerzahl anzupassen.\newline
\newline
\textbf{Hinweis:} Neue bzw. Ersatzanleger m{\"u}ssen bei aktuellem Stand noch
per Hand gesetzt werden.
\newline\newline
\textbf{Skriptname:} cantstoptheflood.py

\section {Trockenheiten}
Eine Trockenheit symbolisiert das Fehlen von Wasser in einem Gebiet. Diese, wenn
nicht schnell beendet, reduziert zeitweise den Ertrag gewisser Produkte. Die
Regeneration der Felder sinkt stetig bis die Trockenheit beendet ist oder eine
Regeneration von 0 erreicht wurde. Dauert die Trockenheit l{\"a}nger an, so
sinkt mit jeder weiteren Auswertungsperiode die Qualit{\"a}t des Feldes. Bei
einer Feldqualit{\"ua}t von anfangs 5, wird es also zusammen 10 Tage dauern bis
sowohl Regeneration wie auch Qualit{\"a}t eines Feldes auff 0 gesunken
sind.\newline 
Ist dieser Punkt erreicht, beginnt das Land sich bei fortschreitender
Trockenheit zu wandeln. Regul{\"a}re Felder werden zu W{\"u}ste. Als
solche sind die meisten Produktionsgeb{\"a}ude nicht mehr in der Lage zu
produzieren, komplette Wirtschaftszweige k{\"o}nnen wegbrechen.
Weiterverarbeitende Gewerbe k{\"o}nnen weiter produzieren.\newline
\newline \textbf{Hinweis:} Es ist dabei zu achten das bei manueller Ausl{o}sung nur Regionen mit passender Ausgangssituation gew{\"a}hlt werden. Der {\"U}bergang von ewigen Winter in eine trockene W{\"u}ste ist nicht immer
glaubw{\"u}rdig.
\newline\newline
\textbf{Skriptname:} dryland.py
\newline\newline
\textbf{Aufrufparameter:} InselID, Dauer, Variante\newline \newline
Werden einer oder mehrere der Parameter nicht gesetzt, wird von vulkan.py ein
zuf{\"a}lliger Wert zugewiesen. Die Parameter Geschwindigkeit, Zielkoordinate
und InselID werden in Abh{\"a}ngigkeit der angegebenen Variante ausgewertet.

\section {Verl{\"a}ngerte Jahreszeiten}
Aufgrund des technischen Aufbaus von Insulae ist es m{\"o}glich f{\"u}r jede Region /
Insel / Kontinent eigene Abl{\"a}ufe der Jahreszeiten zu definieren. Wenn auf
Del'ro Den Sommer herrscht, liegt Akrunia im Winter.
Dieser Mechanismus wurde erweitert und ge{\"o}ffnet. Damit kann die Dauer einer
Jahreszeit pro Insel entweder zuf{\"a}llig per Skript oder manuell {\"u}ber das
Verwaltungsportal verl{\"a}ngert oder verk{\"u}rzt werden.\newline

Die sonst feste Planung, n Tage Fleisch / Getreide / Holz / Steine
vorr{\"a}tig zu halten, kann somit regional hinf{\"a}llig werden. Die Spieler
werden gezwungen sich auf ge{\"a}nderte Rahmenbedingungen einzustellen.
M{\"o}glichkeiten daf{u}r sind unter anderem Raubz{\"u}ge in Regionen welche die
knappen Ressourcen in Masse haben oder verst{\"a}rkter Handel.
\newline\newline
\textbf{Skriptname:} vierjhahreszeiten.py
\newline\newline
\textbf{Aufrufparameter:} InselID, Dauer

\section {Wechselnde Landschaftsformen}
Zum Wechsel einer Landschaftsform in eine andere.
\newline\newline
F{\"u}r dieses Szenario ist bisher noch keine wirkliche Anwendung gefunden.
gefunden.\newline \textbf{Skriptname:} changeforyou.py
\newline\newline
\textbf{Aufrufparameter:} InselID, Dauer, Neue Landart

%****************
% Dokumentenklasse und Packete laden
%________________
\documentclass[10pt,a4paper,openany]{book} 
 
%****************
% define medatata
%________________
\newcommand{\mail}{game.insulae@googlemail.com}
\newcommand{\titlemain}{Dynamische Ereignisse in Browserspielen}
\newcommand{\titlesub}{Am Beispiel von Insulae}
\newcommand{\keywords}{Browserspiele,Insulae}
\newcommand{\genprogram}{Eclipse with Texclipse-Plugin}
\newcommand{\subject}{Einbindung dynamischer Ereignisse in statische
Browsergames}
 
\usepackage{ngerman, fancyhdr, graphicx, makeidx, setspace}
\usepackage[T1]{fontenc}

\PassOptionsToPackage{hyphens}{url} %Umbr�che nach '-' erlauben 
\usepackage[pdfa,breaklinks=true]{hyperref}
\makeindex

\setlength{\headheight}{14.0pt}
\pagenumbering{arabic}
\setlength{\parindent}{0pt}

%\pagestyle{fancy}
%\fancyhead[L] { Premiumfunktionen }
%\fancyfoot[r] { \small{http://www.janhkrueger.de/blog} }

%****************
% Verhalten der Hyperlinks definieren.
%________________
\hypersetup{
    colorlinks,%
    citecolor=black,%
    filecolor=black,%
    linkcolor=black,%
    urlcolor=black,
    pdfstartview={FitH},
    pdftitle=\titlemain,
    pdfauthor=Jan H. Kr{\"u}ger, %Verfasser
    pdfcreator=\genprogram, %Anwendung
    pdfproducer=\genprogram, %Anwendung
    pdfkeywords=\keywords,
    pdfsubject=\subject,
    breaklinks=true,
    baseurl=http://www.janhkrueger.de/browsergames/generisch/Premiumfunktionen.pdf
}

%****************
% Titel und Autor
%________________
\title{ \Huge{\titlemain \\ \small{\titlesub}}}
\author{Jan H. Kr{\"u}ger\thanks{\mail}}
\date{Friedrichsdorf den 23.07.2008}

\setlength{\parindent}{0pt}
\pagenumbering{arabic}
 
\renewcommand{\labelitemi}{--}
 
\setcounter{tocdepth}{1}



\begin{document}
 
\maketitle
\tableofcontents

%****************
% Kapitel Zweck des Dokumentes
%________________
\chapter{Zweck des Dokumentes}
Dieses Dokument dient als Ideensammlung f{\"u}r zufaellige Ereignisse in
Browsergames wie Insulae. Es erhebt kein Anspruch auf Vollst{\"a}ndigkeit oder
Korrektheit sondern dient lediglich als Sammelbecken und Abbarbeitungsliste.
\newline\newline
\textbf{Abgespeckte Version. Sie es als Demo. Als Teaser. Trailer. Werbeclips.
Was auch immer. Das volle Dokument kommt nicht in Spielerh{\"a}nde. Dann wissen
die ja gleich sofort was alles auf sie zukommen kann. Nene, nix da.
Gerade die Questsettings habe ich ausgeblendet da niemandem der Spass
genommen werden soll. Ausserdem habe ich ein paar Stufe 2 Infos nicht im
aktuellen Deploy drin da die Stufe 2 Infos f{\"u}r euch aktuell nicht interessant sein sollten.}
\newline\newline
Grob l{\"a}\ss t sich das Dokument in die folgenden Bereiche einteilen:\newline
\begin{itemize}
\item {Warum Spielwelt-{\"u}bergreiffende Ereignisse wichtig sind.}
\item {Landver{\"a}nderungen}
\item {Zeitlich begrenzte Ereignisse}
\item {Lokale Questen, wiederholbar}
\item {Globale Questen, wiederholbar}
\end{itemize}

Bei den Questen handelt es sich um solche welche durch die Nutzung von Triggern
auf der Welt nach Ausloesung automatisch ablaufen koennen, ohne Eingreifen eines
Supporters. Die Sammlung der organisierten Questen befindet sich in einem
separaten Dokument.


%****************
% Kapitel Landshcaftsveraenderten
%________________
\chapter{Landver{\"a}nderungen}
Unter den Landver{\"a}nderungen werden alle Szenarien zusammen gefasst, welche sich
auf das optische Erscheinungsbild der Karte auswirken.\newline
\newline
Nach aktuellem Stand bietet sich jedes der hier vorgestellten Szenarien auch an
um das Eingreifen der G{\"o}tter in die Welt darzustellen. Es ist noch zu
pr{\"u}fen inwieweit sich einige dieser Szenarien eignen durch m{\"a}chtige
Zauber durch Spieler bewu\ss t herbeigef{\"u} hrt werden k{\"o}nnen.

\section {Vulkanausbr{\"u}che}
Vulkanausbr{\"u}che dienen dazu nachhaltig die Landmasse zu ver{\"a}ndern.
Aufgrund ihrer grossen Zerst{\"o}rungskraft welche potentiell s{\"a}mtliche Geb{\"a}ude
und Landschaftsaufwertungen vernichten kann, ist dieses Ereigniss nur {\"a}usserst
selten zu triggern.
\linebreak\linebreak
Bei Verwendung von Vulkanausbr{\"u}chen ist zu Beginn jeder Runde klar zu stellen
das es zu solchen kommen kann mit einem Hinweis darauf was fuer endgueltige
Folgen ein Vulkanausbruch mit sich bringt. Das Gebiet in direkter N{\"a}he des
Vulkanes wird unmittelbar zerstort. Daraufliegende Geb{\"a}ude und Stra\ss en
werden ebenfalls unwiederbringlich zerst{\"o}rt. Es findet kein materieller oder
finanzieller Ausgleich statt. Charaktere welche zum Zeitpunkt des Ausbruches
sich in dem Gebiet aufhalten, werden zum n{\"a}chsten Tempel transferiert.
Sollte sich in der Region des Ereignisses ein Tempel befinden, ist dieser mit
Beginn des Ereignisses zu deaktivieren und sterbende Charaktere auf umliegende
Tempel zu verteilen.\linebreak

Nach dem Ausbruch gibt es mehrere Varianten wie das Ereignis fortgef{\"u}hrt
werden kann.\newline
- Ende des Ausbruches\newline
- Lava fliesst gen K{\"u}ste\newline
- Vernichtung der Insel\newline

Bei Variante 1, Ende des Ausbruches ist weiter nichts mehr zu beachten.  Die
vernichteten L{\"a}ndereien werden als Steinfelder fortgef{\"u}hrt welche wie
gehabt von Charakteren bewirtschaftet werden koennen. Das zentrale Feld besitzt
eine erh{\"o}hte Wahrscheinlichkeit Quelle von Edelsteinen zu sein.\newline
\linebreak
Sollte die Lava wie in Variante 2 gen K{\"u}ste flie\ss en, muss der Vektor zu
Beginn festgelegt werden. Von Vulkankrater bis zur K{\"u}ste wird ein 3-Felder
breiter Streifen Land vollst{\"a}ndig zerst{\"o}rt. Der Lavastrom breitet sich
mit mindestens einem Feld pro Tag aus. Schnelleres \glqq flie\ss en\grqq  der
Lava ist m{\"o}glich bis zu einer Rate von einem Feld pro Stunde.\newline
\linebreak
Die vollst{\"a}ndige Vernichtung einer Insel schlie\ss lich ist vergleichbar mit
Variante 2. Der Unterschied besteht lediglich darin das die Lava in alle
Richtungen in sich konzentrisch um den Startpunkt ausbreitet. Diese Variante ist
nur zu empfehlen wenn eine Insel vergr{\"o}\ss ert oder gegen eine andere
Inselvorlage ausgetauscht werden soll. In diesem Fall wird nachdem die Lava
erkaltet ist, anstelle der bisherigen Insel die neue entstehen. Ich empfehle um
einen glaubhaften {\"U}ebergang zu gestalten eine Kombination aus Vulkanausbruch
und {\"U}berflutung. Bei der {\"U}berflutung ist jedoch die {\"u}bliche Eintragung
in der {\"o}ffentlich zug{\"a}nglichen Historie auszublenden.\newline
\newline
Vulkanausbr{\"u}che k{\"o}nnen in einer laufenden Runde auch genutzt werden um
die Landmasse der Spielerzahl anzupassen. Gleichzeitig dienen sie dazu den
Eindruck einer ver{\"a}nderlichen Welt darzustellen.\newline
\newline
\textbf{Skriptname:} ringoffire.py
\newline\newline
\textbf{Aufrufparameter:} Startkoordinate, Variante, Geschwindigkeit,
Zielkoordinate, InselID\newline \newline
Werden einer oder mehrere der Parameter nicht gesetzt, wird von ringoffire.py
ein zuf{\"a}lliger Wert zugewiesen. Die Parameter Geschwindigkeit, Zielkoordinate
und InselID werden in Abh{\"a}ngigkeit der angegebenen Variante ausgewertet.

\section {{\"U}berflutungen}
Eine {\"U}berflutung ist in ihrer Auswirkung das Gegenteil zu
Vulkanausbr{\"u}chen. Anstatt Land zu schaffen, wird
hier Land genommen.\newline Daher treffen die oben genannten Anmerkungen in
verleichbarer Form auch auf {\"U}berflutungen zu. So kann eine {\"U}berflutung ebenfalls dazu genutzt werden die Landmasse der Spielerzahl anzupassen.\newline
\newline
\textbf{Hinweis:} Neue bzw. Ersatzanleger m{\"u}ssen bei aktuellem Stand noch
per Hand gesetzt werden.
\newline\newline
\textbf{Skriptname:} cantstoptheflood.py

\section {Trockenheiten}
Eine Trockenheit symbolisiert das Fehlen von Wasser in einem Gebiet. Diese, wenn
nicht schnell beendet, reduziert zeitweise den Ertrag gewisser Produkte. Die
Regeneration der Felder sinkt stetig bis die Trockenheit beendet ist oder eine
Regeneration von 0 erreicht wurde. Dauert die Trockenheit l{\"a}nger an, so
sinkt mit jeder weiteren Auswertungsperiode die Qualit{\"a}t des Feldes. Bei
einer Feldqualit{\"ua}t von anfangs 5, wird es also zusammen 10 Tage dauern bis
sowohl Regeneration wie auch Qualit{\"a}t eines Feldes auff 0 gesunken
sind.\newline 
Ist dieser Punkt erreicht, beginnt das Land sich bei fortschreitender
Trockenheit zu wandeln. Regul{\"a}re Felder werden zu W{\"u}ste. Als
solche sind die meisten Produktionsgeb{\"a}ude nicht mehr in der Lage zu
produzieren, komplette Wirtschaftszweige k{\"o}nnen wegbrechen.
Weiterverarbeitende Gewerbe k{\"o}nnen weiter produzieren.\newline
\newline \textbf{Hinweis:} Es ist dabei zu achten das bei manueller Ausl{o}sung nur Regionen mit passender Ausgangssituation gew{\"a}hlt werden. Der {\"U}bergang von ewigen Winter in eine trockene W{\"u}ste ist nicht immer
glaubw{\"u}rdig.
\newline\newline
\textbf{Skriptname:} dryland.py
\newline\newline
\textbf{Aufrufparameter:} InselID, Dauer, Variante\newline \newline
Werden einer oder mehrere der Parameter nicht gesetzt, wird von vulkan.py ein
zuf{\"a}lliger Wert zugewiesen. Die Parameter Geschwindigkeit, Zielkoordinate
und InselID werden in Abh{\"a}ngigkeit der angegebenen Variante ausgewertet.

\section {Verl{\"a}ngerte Jahreszeiten}
Aufgrund des technischen Aufbaus von Insulae ist es m{\"o}glich f{\"u}r jede Region /
Insel / Kontinent eigene Abl{\"a}ufe der Jahreszeiten zu definieren. Wenn auf
Del'ro Den Sommer herrscht, liegt Akrunia im Winter.
Dieser Mechanismus wurde erweitert und ge{\"o}ffnet. Damit kann die Dauer einer
Jahreszeit pro Insel entweder zuf{\"a}llig per Skript oder manuell {\"u}ber das
Verwaltungsportal verl{\"a}ngert oder verk{\"u}rzt werden.\newline

Die sonst feste Planung, n Tage Fleisch / Getreide / Holz / Steine
vorr{\"a}tig zu halten, kann somit regional hinf{\"a}llig werden. Die Spieler
werden gezwungen sich auf ge{\"a}nderte Rahmenbedingungen einzustellen.
M{\"o}glichkeiten daf{u}r sind unter anderem Raubz{\"u}ge in Regionen welche die
knappen Ressourcen in Masse haben oder verst{\"a}rkter Handel.
\newline\newline
\textbf{Skriptname:} vierjhahreszeiten.py
\newline\newline
\textbf{Aufrufparameter:} InselID, Dauer

\section {Wechselnde Landschaftsformen}
Zum Wechsel einer Landschaftsform in eine andere.
\newline\newline
F{\"u}r dieses Szenario ist bisher noch keine wirkliche Anwendung gefunden.
gefunden.\newline \textbf{Skriptname:} changeforyou.py
\newline\newline
\textbf{Aufrufparameter:} InselID, Dauer, Neue Landart


%****************
% Zeitlich begrenzte Ereignisse
%________________
\chapter{Zeitlich begrenzte Ereignisse}
Unter die im Titel genannten Ereignisse wird all jenes eingereiht, was \newline
TODO: Beschreibung noch nicht fertig. Viel zu sp{\"a}t nun.

\section {Auftreten besonderer Rohstoffe}
F{\"u}r einen begrenzten Zeitraum entstehen auf zuf{\"a}llig ausgew{\"a}hlten
leeren Feldern seltene Hotspots sonst nicht vorkommender Rohstoffe. Prinzipiell
kann mittels Aufruf aus dem Verwaltungsportal jeder Rohstoff so platziert
werden. Es ist jedoch zu beachten das nicht f{\"u}r jeden Rohstoff entsprechende
Bilder vorliegen. Von der Glaubw{\"u}rdigkeit mancher Kombinationen einmal ganz
abgesehen.
\newline\newline \textbf{Skriptname:} gimmegoods.py \newline\newline
\textbf{Aufrufparameter:} InselID, RohstoffID, Dauer
\newline\newline
Wird kein Parameter wird angegeben, verarbeitet das Skript jede Insel und
w{\"a}hlt zuf{\"a}llig Rohstoff und Dauer aus. Hierbei wird die hinterlegte
Positivliste angewendet.

\section {Wandernde H{\"a}ndler}
Innerhalb der Welt von Insulae gibt es H{\"a}ndler welche nicht fest an einen
Ort festgelegt sind, sondern im Lauf der Zeit ueber die Inseln wandern. Diese
Wanderungen werden in regelm{\"a}\ss igen Abst{\"a}nden ausgef{\"u}hrt. Eine
manuelle Wanderung ist hierbei m{\"o}glich.\newline\newline
Wandernde H{\"a}ndler bieten die Chance bestimmte Rohstoffe / Rezepte,
Gegenst{\"a}nde mit einem Seltenheitsfaktor zu versehen. Diese
H{\"a}ndler k{\"o}nnen in jedem Gebiet auftauchen, auch wenn
dieses f{\"u}r manche Spieler nicht erreichbar ist. Es liegt hier
bei den Spielern, wie nach Auftauchen eines H{\"a}ndlers
verfahren wird, ob der aktuelle Ort geheim bleibt oder die
Mitspieler darauf hingewiesen werden.\newline
Sollte der H{\"a}ndler bek{\"a}mpft und besiegt werden, so startet dieser nach
Ablauf einer Woche an einem zuf{\"a}lligen Puntk seiner Route. Die Koordinate
von welcher er vertrieben wurde, wird jedoch f{\"u}r ein halbes Jahr
deaktiviert. Wer den H{\"a}ndler vor die T{\"u}r setzt, muss eben nicht damit
rechnen das dieser so schnell wieder kommt.
\newline
Im Gegenzug kann es auch vorkommen, das Diebesbanden von Insel zu Insel ziehen
und dabei entsprechenden wirtschaftlichen Schaden anrichten indem Rohstoffe aus
Geb{\"a}uden verschwinden, Pferde aus Routen gestohlen werden etc.
Diese Art von wandernden H{\"a}ndlern kann jedoch nach
Entdeckung bek{\"a}mpt werden. Der betreffende H{\"a}ndler
starten dann seine Reise an einem zuf{\"a}llig ausgew{\"a}lten
Ort seiner Route nach einer Woche neu. Auf der betreffenden
Insel wird allerdings fuer ein halbes Jahr ein weiterer Punkt
erzeugt an dem diese Art von H{\"a}ndler erscheinen kann.\newline\newline
\textbf{Skriptname:}
gypsiestrampsandthieves.py
\newline\newline
\textbf{Aufrufparameter:} HaendlerID, Rotationen
\newline\newline Wird kein Parameter wird angegeben, verarbeitet das Skript jede Insel und w{\"a}hlt zuf{\"a}llig Rohstoff und Dauer aus. Hierbei wird die hinterlegte Positivliste angewendet.

\section {Ver{\"a}ndertes Monsterwachstum}
Das Monsterwachstum auf einer Insel wird stimuliert oder ausgebremst. Eine
Stimulation kann daf{\"u}r sorgen das die Spieler sich verst{\"a}rkt um die
Monster k{\"u}mmern m{\"u}ssen. Muss nur entsprechend weh tun. \newline Sofern
nichts anderes vorgegeben wird, werden die inden t{\"a}glichen Datenerhebungen
gesammelten Daten bez{\"u}glich Anzahl und St{\"a}rke der Spieler zur Bestimmung
von St{\"a}rke und Dauer des ver{\"a}nderten Wachstums herangezogen.
\newline\newline
\textbf{Skriptname:}
wewillnotgrowold.py 
\newline\newline
\textbf{Aufrufparameter:} InselID, MonsterID, Variante, Dauer
\newline\newline
Die Variante gibt an, ob es zu einem verst{\"a}rktem oder geringerem Wachstum
kommt. Wird die Variante nicht gesetzt, wird von einem Wachstum ausgegangen.


\renewcommand{\indexname}{Stichwortverzeichnis}

\end{document}
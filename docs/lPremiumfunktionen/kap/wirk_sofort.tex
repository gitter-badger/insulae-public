\chapter{Sofortige Auswirkungen}
Diese Gegenst�nde stehen sofort zur Verf�gung nachdem sie erworben wurden und
verschwinden meist nach der Benutzung. Sie k�nnen mehrfach erworben
werden.\\
Bisher sind noch keine Vorschl�ge gemacht worden damit Spieler direkte Vorteile
im Kampf gegen andere Spieler erhalten. Solches k�nnen Waffen und R�stungen
sein oder auch einzigartige Zauberspr�che.

\section {Umbennenung eines Schiffes}\index{Schiffname}
Hiermit wird ein Spieler in die Lage versetzt eines seiner Schiffe, und zwar
exakt eines, mit einem individuellen Namen zu versehen.

\section {Umbenennung eines NPCs}\index{NPC-Name}
Ein einziger NPC eines Charakters, unabh�ngig von dessen Aufenhaltsort,
kann mit einem neuen Namen versehen werden. Dieses Objekt kann mehrfach
erworben werden, um so zum Beispiel die gesamte Laufgruppe anzupassen. Der
Hauptcharakter kann hiervon nicht profitieren.

\section {Umbenennung des Gladiators}\index{Gladiator}
Wenn der Hauptcharakter umenannt wurde, beh�lt der Gladiator den alten
Namen. Hiermit wird ein Spieler in die Lage versetzt den Namen des Gladiators
auf den aktuellen HC-Namen zu setzen.
 
\section {Reduzierung der Bauzeiten um einen Tag f�r ein
Geb�ude}\index{Bauzeit}\index{Kaufobjekt} Mit Hilfe dieser Transaktion kann
die Bauzeit eines Geb�udes um einen Tag reduziert werden. Ein Geb�ude von nur einem Tag Bauzeit ist damit sofort
fertig gestellt, eines mit zweien nach einem etc. Pro Geb�ude kann diese
Transaktion nur einmal angewendet werden.

\section {Kauf eines Premium-Schiffes}\index{Schiffkauf}\index{Kaufobjekt}
Spezielle Schiffe stehen im Premiumshop bereit. Diese unterscheiden sich von
den regul�ren das sie h�here Reichweite, mehr Laderaum aufweisen oder
schneller durch die See fahren.
 
\section {Kauf eines Premium-Lasttieres}\index{Tierkauf}\index{Kaufobjekt}
Spezielle Lasttiere stehen im Premiumshop bereit. Diese weisen eine h�here
Tragf�higkeit oder geringe AP-Einbu\ss en auf wie die regul�r in Insulae
erh�ltlichen Lasttiere.
 
\section {Kauf von Wohnrecht}\index{Wohnrecht}\index{Kaufobjekt}
Wenn ein Spieler seinem Charakter ein Wohnhaus g�nnen will jedoch nicht die
entsprechende Ehrung durch einen Konvent erh�lt, kann das Wohnrecht auch
gekauft werden. Es wird bei der Anzeige der Wohnung ein kleines Icon
dargestellt damit zu erkennen ist woher das Wohnrecht stammt.

\section {Rassenwechsel eines NPCs}\index{Rassenwechsel}\index{Kaufobjekt}
Ein frisch angeworbener NPC kann nachtr�glich zu einem Angeh�rigen einer
anderer Rasse umgestellt werden.\\
Dies ist nur m�glich wenn der NPC insgesamt zwischen 0 und 1000
Erfahrungspunkte bisher besitzt.

\section {Rassenwechsel}\index{Rassenwechsel}\index{Kaufobjekt}
Hiermit kann die Rasse eines Charakters, auch des Hauptcharakters gewechselt
werden. Es stehen als Zielrasse auch Rassen der Kategorie 2, jedoch nicht der
Kategorie 3 und h�her zur Verf�gung und damit auch Rassen welche nicht
bei der Charaktererschaffung gew�hlt werden k�nnen.\\
Dies kann zu jeder Zeit getan werden und hat daher h�her gewertet zu werden
wie der Rassenwechsel eines NPCs.\\
Ein hiermit gelegter Charakter erh�lt f�r einen Monat einen geringeren Gewinn
von Erfahrungspunkten in H�he von 20\%.